\documentclass[letterpaper,10pt]{book}
\usepackage[utf8]{inputenc}

% Set the margins and text area to comply with university guidelines
% Use the geometry package to accomplish this. Binding edge must be
% 40mm, other edges greater than 20mm (remember header/footer are outside
\usepackage[
  paper=letterpaper,
  lmargin=1in,
  rmargin=1in,
  tmargin=1in,
  bmargin=1in]{geometry}

% Use symbols for footnotes rather than superscript numbers
\renewcommand\thefootnote{\fnsymbol{footnote}}

% Modify the captions so that they look better, the way I like them at least...
\usepackage[font={small,bf},textfont=md,format=hang,margin=20pt,singlelinecheck=false]{caption}

% Use the extra list environments available
\usepackage{paralist}

% Now to get better table handling
\usepackage{tabularx}

% Use the AMS packages for better mathematical typesetting
\usepackage{amsmath}
\usepackage{amsfonts}
\usepackage{amssymb}

%\usepackage[dvips,final]{graphicx}
\usepackage[pdftex,final]{graphicx}
\usepackage[pdftex]{color}

% Use the varioref package for fancy references, \vref and friends
\usepackage{varioref}

% Program listings need to be typeset
\usepackage[final]{listings}
\lstloadlanguages{C++}
\lstset{language=C++, captionpos=b, basicstyle=\footnotesize, commentstyle=\scriptsize, breaklines=true, showstringspaces=false, tabsize=2, xleftmargin=5mm, xrightmargin=5mm}

\lstnewenvironment{cpplst}[1][]
  {\lstset{language=C++, captionpos=b, basicstyle=\footnotesize, commentstyle=\scriptsize, breaklines=true,xleftmargin=5mm, xrightmargin=5mm, #1}
  \singlespacing}
  {\doublespacing}

% Make fancy headers and everything...
\usepackage{fancyhdr}
\pagestyle{fancy}
% Headers - left, center and right.
\lhead{Practicing Open Science}
\chead{DRAFT}
\rhead{Kitware, Inc.}
% Footers - as before
%\lfoot{\scriptsize \emph{\svnInfoFile}}
\cfoot{\small \thepage}
%\rfoot{\scriptsize \emph{Rev: \svnInfoRevision , \svnInfoDate}}
\renewcommand{\headrulewidth}{0.5pt}
\renewcommand{\footrulewidth}{0.5pt}

%\addtolength{\headheight}{0.5pt} % make space for the rule
%\setlength{\headheight}{15pt}
\fancypagestyle{plain}{%
  \fancyhead{} % get rid of headers on plain pages
  \renewcommand{\headrulewidth}{0pt} % and the line
}

% Customise float placement
\setcounter{topnumber}{4}
\setcounter{totalnumber}{5}
\renewcommand\topfraction{0.8}
\renewcommand\floatpagefraction{0.7}

\title{\textbf{Practicing Open Science:
Changing the World Through Sharing Everything}}
\date{\today}

% Now to define several macros for commonly used abbreviations and such
\usepackage{xspace}
\newcommand{\eg}{\emph{e.g.},\xsapce}
\newcommand{\ie}{\emph{i.e.},\xspace}
\newcommand{\etc}{etc.\@\xspace}

\newcommand{\etal}{\emph{et al}\xspace}

% Use the cite package to better organise citations
\usepackage[superscript]{cite}
% Put [] around the superscript citations
\makeatletter
\def\@citess#1{\textsuperscript{[#1]}}

\usepackage{url}
% Use in the form \url{http://mysite.co.uk/~marcus/} or \url!http://mysite.co.uk/~marcus/!

\usepackage{colortbl}
% 21st century - make some real links in the PDF, but keep the colors reasonable...
\usepackage[colorlinks=true,urlcolor=blue,citecolor=black,linkcolor=black,final=true]{hyperref}

\begin{document}

\maketitle

\newpage

\tableofcontents

\newpage

\section*{Foreword}

Brief foreward introducing the book.

\chapter{Introduction}

\section{Why Open Science?}
% Motivation
% Who we are as authors, our commitment to open science

\section{Purpose of the Book}
% Show how to do Open Science in a practical way

\section{Organization of the Book}
% Three parts, multiple chapters per part

\section{Acknowledgements}
% Communities, OS leaders, scientists and practitioners, Kitware



\part{The Imperative of Open Science}

% This psrt addresses the broader impact of open science.

\chapter{Overview}
% Provide a general overview of the arguments to this firt Part I

\section{Science: Theory, Experiment, Computation}

\section{The Role of Software, Data and Publication}


\chapter{History}
% History Lesson: Community principles, social networking, communication

\section{A History of Collaboration}
% history of science, how societies evolved}

\section{Growing communities in the Digital World}
%

\section{Communication Channels}
% (blog, wiki, IRC, twitter, etc.) [Marcus}

\section{Community details}
% (G+, Twitter, IRC, IM, mailing lists, wiki, forum, stackoverflow)


\chapter{Innovation}
% Solving the World’s Problems

\section{Increasing Complexity}
% requires teams

\section{Leveraging Expertise}
% Leveraging the micro-expertise of people (Networked Science)


\chapter{Research}
% Problems in research and how OS addresses them

\section{Reproducibility}
% Validation, doing science (e.g., chemistry melting points) [Marcus]


\chapter{Publishing}
% Publishing, Science, PLoS, Princeton/Yale

\section{Institutional Repositories}

\section{Archives}
% ArXiv [Marcus]
% Archives


\chapter{Education}
% All things related to education

\section{Open Courseware}

\section{DIY}
% Roll your own. DIY, empower people to do it themselves

\section{The Learning Experience}
% K-12, STEM, College

\section{Outreach}
% (GSoC, community involvement, etc.) [Marcus]


\chapter{Intellectual Property}
% The Misnomer of Intellectual Property

\section{Patents, Copyright, other IP}
% Harm to science, harm to data
% The mistakes as copyrightable material, cannot copyright facts
% Bob Doyle Patent Act ? / University patents
% The ineffectiveness of Licensing Operations in the university
% The problem of tenure

\section{Data}
%

\section{Software}
%

\section{Publications}
%

\section{Access / Copyright Assignment}
%

\section{Commercial Access}
% Refer to non-commercial sub-section below, the important of commercial
% access (e.g., Google translates a paper for you, computer-based data
% mining)





\chapter{FundingModels}
% Different approaches Agencies
% EPSRC
% Wellcome Trust
% NIH Open Access
% DoD


\chapter{AcademicTensions}
% Reward mechanisms
% Data and software impacts
% Publications impacts


\chapter{CommercialTensions}
% The art of throwing money in the right way [Marcus]
% Google Summer of Code [Marcus]
% KDE, Qt - The relationship of community to business, dual license [Marcus]
% Boost - community support of C++ standard [Marcus]
% Non-commercial use / free for academics [Marcus]


\chapter{BusinessModels}
% Various  models
% Making money off of open source



\part{The Way of the Source}

%Show how to do it in detail

\input{Details-Reproducibility}

\chapter{DVCS}
% software process, testing, review (Hello World) [Marcus]

% Things that pass the mathematics test: ( If the language of science has
% changed from mathematics (which anyone could write and read), to code, the
% same everyone must apply for it to work.) Cameron Neylon - programming is
% becoming the new mathematics of science

% This is why patents should not be applicable to software, algorithms, basic procedures
%% Jackson Lab patent infringement example

% CMake gives the ability to compile code on all the major platforms.

% Python code can be run on all major platforms

% Java code

% Platform Agnostic
%% Build systems




\chapter{DataCentricComputing}
% The data management issue. Data ia a fourth paradigm?

% Databases and their types.

% Data management in HPC setting (the challenge of IO as we head to exascale)

% How to host data

% Move computation to the data


\chapter{FlexibleTools}
% Open Frameworks

% Interpreted languages (e.g., Python)


\chapter{Licensing}
% Practical effects of choosing various licenses


\chapter{HelloWorld}
%  reproducibility, building up a system - build, add Midas fetch, Qt, Boost, VTK, database access...


\chapter{Challenges}
% Who hosts the data?
% Who pays for the computation?
% The Role of Publishers


\chapter{Scaling}
% Scaling effects on community size
% Community governance





\part{Case Studies}

% Each case study should demonstrate an aspect of the Open Science
% story. Each chapter is not a detailed technical document, but rather a
% demonstrates a quality of OS.

\chapter{CMake}
%enabler of cross-platform development tools



\chapter{MIDAS/Journal}
%all things data centric

\section{ParaViewWeb}

\section{MongoDB}


\chapter{VTK}
% ground-up community building


\chapter{ITK}
% top-down community building




\chapter{ParaView}
% HPC Research Platform, collaboration with National Labs


\chapter{Titan}
%how a community formed by extending a basic tool (VTK) and developing
%supporting technology. [Marcus]


\chapter{Avogadro/Blue Obelisk}
%-> Open Chem - creation of a broad framework


\chapter{KDE/Qt}
% Not sure we can do this chapter justice. Originally had it on the book
% but then had second thoughts.


\chapter{Boost}
% Community response to a slow and deficient C++ standard


\chapter{3D Slicer}
% Biomedical research done the open source way
% Open Source Reseource Program
% Community formation, hackathons, funding models
% reusing open source tools, giving credit



\end{document}
