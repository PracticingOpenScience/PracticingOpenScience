\documentclass[letterpaper,10pt]{book}
\usepackage[utf8]{inputenc}

% Set the margins and text area to comply with university guidelines
% Use the geometry package to accomplish this. Binding edge must be
% 40mm, other edges greater than 20mm (remember header/footer are outside
\usepackage[
  paper=letterpaper,
  lmargin=1in,
  rmargin=1in,
  tmargin=1in,
  bmargin=1in]{geometry}

% Use symbols for footnotes rather than superscript numbers
\renewcommand\thefootnote{\fnsymbol{footnote}}

% Modify the captions so that they look better, the way I like them at least...
\usepackage[font={small,bf},textfont=md,format=hang,margin=20pt,singlelinecheck=false]{caption}

% Use the extra list environments available
\usepackage{paralist}

% Now to get better table handling
\usepackage{tabularx}

% Use the AMS packages for better mathematical typesetting
\usepackage{amsmath}
\usepackage{amsfonts}
\usepackage{amssymb}

%\usepackage[dvips,final]{graphicx}
\usepackage[pdftex,final]{graphicx}
\usepackage[pdftex]{color}

% Use the varioref package for fancy references, \vref and friends
\usepackage{varioref}

% Program listings need to be typeset
\usepackage[final]{listings}
\lstloadlanguages{C++}
\lstset{language=C++, captionpos=b, basicstyle=\footnotesize, commentstyle=\scriptsize, breaklines=true, showstringspaces=false, tabsize=2, xleftmargin=5mm, xrightmargin=5mm}

\lstnewenvironment{cpplst}[1][]
  {\lstset{language=C++, captionpos=b, basicstyle=\footnotesize, commentstyle=\scriptsize, breaklines=true,xleftmargin=5mm, xrightmargin=5mm, #1}
  \singlespacing}
  {\doublespacing}

% Make fancy headers and everything...
\usepackage{fancyhdr}
\pagestyle{fancy}
% Headers - left, center and right.
\lhead{Practicing Open Science}
\chead{DRAFT}
\rhead{Kitware, Inc.}
% Footers - as before
%\lfoot{\scriptsize \emph{\svnInfoFile}}
\cfoot{\small \thepage}
%\rfoot{\scriptsize \emph{Rev: \svnInfoRevision , \svnInfoDate}}
\renewcommand{\headrulewidth}{0.5pt}
\renewcommand{\footrulewidth}{0.5pt}

%\addtolength{\headheight}{0.5pt} % make space for the rule
%\setlength{\headheight}{15pt}
\fancypagestyle{plain}{%
  \fancyhead{} % get rid of headers on plain pages
  \renewcommand{\headrulewidth}{0pt} % and the line
}

% Customise float placement
\setcounter{topnumber}{4}
\setcounter{totalnumber}{5}
\renewcommand\topfraction{0.8}
\renewcommand\floatpagefraction{0.7}

\title{\textbf{Practicing Open Science:
Changing the World Through Sharing Everything}}
\date{\today}

% Now to define several macros for commonly used abbreviations and such
\usepackage{xspace}
\newcommand{\eg}{\emph{e.g.},\xsapce}
\newcommand{\ie}{\emph{i.e.},\xspace}
\newcommand{\etc}{etc.\@\xspace}

\newcommand{\etal}{\emph{et al}\xspace}

% Use the cite package to better organise citations
\usepackage[superscript]{cite}
% Put [] around the superscript citations
\makeatletter
\def\@citess#1{\textsuperscript{[#1]}}

\usepackage{url}
% Use in the form \url{http://mysite.co.uk/~marcus/} or \url!http://mysite.co.uk/~marcus/!

\usepackage{colortbl}
% 21st century - make some real links in the PDF, but keep the colors reasonable...
\usepackage[colorlinks=true,urlcolor=blue,citecolor=black,linkcolor=black,final=true]{hyperref}

\begin{document}

\maketitle

\newpage
\tableofcontents
\newpage

\section*{Foreword}

Brief foreward introducing the book.

\chapter{Introduction}

\section{Why Open Science?}

\section{Purpose}

\section{Organization}

\section{Acknowledgements}




\chapter{Broad}

This chapter addresses the broader impact of open science.

\section{Education, Open Courseware}

\section{Funding}

\section{Publishing}

\section{Access/Copyright}

\section{Community Principles, Social Networking, Communication}


\chapter{Details}

Show how to do it in detail

\section{DVCS}

\subsection{Software Process}

\subsection{Testing}

\subsection{Code Review}

\subsection{Hello World}

\section{Reproducibility}

\section{Build System}

\section{Building Community}

Networking, Twitter, Identica, Google+, IRC, mailing lists, wiki, forums, stackoverflow, blogs.

\section{Open Source, Open Data, Open Access---Sharing}

\section{Licensing}

\section{Hello World}

Reproducibility, building up a system, version control, add build system, testing, MIDAS data,
Qt, Boost, VTK, database, network.


\part{Case Studies}

% Each case study should demonstrate an aspect of the Open Science
% story. Each chapter is not a detailed technical document, but rather a
% demonstrates a quality of OS.

\chapter{CMake}
%enabler of cross-platform development tools



\chapter{MIDAS/Journal}
%all things data centric

\section{ParaViewWeb}

\section{MongoDB}


\chapter{VTK}
% ground-up community building


\chapter{ITK}
% top-down community building




\chapter{ParaView}
% HPC Research Platform, collaboration with National Labs


\chapter{Titan}
%how a community formed by extending a basic tool (VTK) and developing
%supporting technology. [Marcus]


\chapter{Avogadro/Blue Obelisk}
%-> Open Chem - creation of a broad framework


\chapter{KDE/Qt}
% Not sure we can do this chapter justice. Originally had it on the book
% but then had second thoughts.


\chapter{Boost}
% Community response to a slow and deficient C++ standard


\chapter{3D Slicer}
% Biomedical research done the open source way
% Open Source Reseource Program
% Community formation, hackathons, funding models
% reusing open source tools, giving credit



\section*{References}

\part{Appendices}
% A collection of miscellaneous material

\chapter{Rants}
% Collect here more volatile, emotional arguments


\chapter{More Information}
% Additional useful information


\chapter{Exercises}
% Put the readers to work


\chapter{Additional Resources}
% General reading and resources.



\end{document}
