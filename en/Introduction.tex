\chapter{Introduction}

\section{Why Open Science?}
% Motivation

In the March 2012 issue of Nature journal, Glenn Begley and Lee Ellis
reported\footnote{\url{http://www.ncbi.nlm.nih.gov/pubmed/22460880}} that in
their attempt to replicate 53 ``landmark'' publications -- papers in top
journals, from reputable labs -- 47 of them could not be
replicated~\cite{Begley2012}.

These were key articles that have driven the roadmap of cancer research in
recent decades. The fact that their finding could not be replicated by an
independent researcher bring into question the core assumptions that many of us
make about the practices of modern science. This lack of reproducibility came
at the cost of millions of dollars of social investment, as well as the cost of
lives lost either directly by the misguided application of inadequate medical
procedures, or indirectly by the eventual delay that these publications
introduced in the actual finding of effective treatements for health conditions
by misleading the research establishment into dead end roads.

Since the time of Galileo Galilei it has been a cornerstone of the scientic
process that the results that are reported must be reproducible by others under
a variety of conditions in order for those results to be accepted as part of
our understanding of the natural world. Unfortunately these fundamental
practice of verifying reproducibility on a regular basis has faded in the daily
practice of many research institutions.

This book is about how to bring back to our daily routine the practices that
materialize the core values of the scientific research method.

Inspired by the motto of the Royal Society\\

\emph{Nullis in Verbia}

Meaning\\

\emph{Do not take anybody's word for it}

The authors describe how tools that are available to all can be used today to
reinstantiate the core value of reproducibilty verification by building on the
cornerstone of Openness, generous sharing of information, reduction and
sometimes full elimination of the barriers that intellectual property pose to
the effective and efficient dissemination of scientific information.

This is a practical book for practical people who hold in high steem the
principles of scientific research and want to restore its practice to
realize the promise of progress and development that is due to society at large.


% Who we are as authors, our commitment to open science

\section{Purpose of the Book}
% Show how to do Open Science in a practical way

\section{Organization of the Book}
% Three parts, multiple chapters per part

\section{Acknowledgements}
% Communities, OS leaders, scientists and practitioners, Kitware


